\documentclass[conference]{IEEEtran}
\usepackage{cite}
\usepackage{amsmath,amssymb,amsfonts}
\usepackage{algorithmic}
\usepackage{graphicx}
\usepackage{textcomp}
\usepackage{xcolor}
\usepackage{hyperref}
\usepackage{booktabs}

\begin{document}

\title{Machine Learning-Based QSAR Modeling for Predicting Bioactivity of Beta-Amyloid A4 Protein Inhibitors: A Computational Approach to Alzheimer's Drug Discovery}

\author{
    \IEEEauthorblockN{Sai Likhith Kanuparthi}
    \IEEEauthorblockA{\textit{Independent Researcher} \\
    Houston, Texas, USA \\
    sailikhithcse@gmail.com}
    \and
    \IEEEauthorblockN{Krishna Jaswanth Boddupalli}
    \IEEEauthorblockA{\textit{PhD Student, Dept. of Biotechnology} \\
    \textit{Vikrama Simhapuri University}\\
    Nellore, Andhra Pradesh, India}
    \and
    \IEEEauthorblockN{Sai Yasaswini}
    \IEEEauthorblockA{\textit{Independent Researcher} \\
    Houston, Texas, USA}
}

\maketitle

\begin{abstract}
Alzheimer's disease is a progressive neurodegenerative disorder affecting over 55 million people worldwide, with beta-amyloid protein aggregation representing a primary therapeutic target. Traditional drug discovery requires over a decade and billions of dollars per approved compound, necessitating computational approaches to accelerate candidate identification. This study develops quantitative structure-activity relationship models using machine learning to predict the bioactivity of small molecule inhibitors against beta-amyloid A4 protein. Bioactivity data for 1,319 compounds were retrieved from the ChEMBL database, filtered for half maximal inhibitory concentration measurements in nanomolar units, and classified as active, intermediate, or inactive based on standard potency thresholds. PubChem molecular fingerprints comprising 881 binary features were calculated and reduced to 178 features via variance threshold filtering. Exploratory analysis confirmed statistically significant differences between active and inactive compounds across all Lipinski descriptors. Thirty-five regression algorithms were benchmarked, followed by five-fold cross-validation and hyperparameter optimization. The optimized random forest and histogram gradient boosting regressors achieved the best predictive performance with coefficient of determination values of 0.78 and root mean squared error of 0.58 on the held-out test set. These results demonstrate practical utility for virtual screening of novel drug candidates and provide a reproducible computational framework for accelerating early-stage Alzheimer's drug discovery.
\end{abstract}

\begin{IEEEkeywords}
Alzheimer's disease, quantitative structure-activity relationship, machine learning, beta-amyloid A4, drug discovery, random forest
\end{IEEEkeywords}

\section{Introduction}

Alzheimer's disease is a chronic neurodegenerative disorder and the leading cause of dementia, affecting approximately 55 million people globally \cite{ref1}. The disease is characterized by the accumulation of extracellular amyloid-beta plaques and intracellular neurofibrillary tangles, which lead to synaptic failure and neuronal death \cite{ref2}. The amyloid hypothesis posits that the aggregation of beta-amyloid peptides, derived from the proteolysis of the amyloid precursor protein, is the primary driver of pathogenesis \cite{ref3}. Consequently, the beta-amyloid A4 protein remains a critical target for therapeutic intervention.

Despite decades of research, the attrition rate for Alzheimer's disease drug candidates remains exceptionally high. Traditional drug discovery pipelines are resource-intensive, often requiring over a decade and approximately \$2.2 billion to bring a new molecular entity to market \cite{ref4}. Computational methods, specifically quantitative structure-activity relationship (QSAR) modeling, offer a cost-effective alternative by predicting the biological activity of compounds based on their chemical structure \cite{ref5}.

This study leverages machine learning to construct robust QSAR models for beta-amyloid A4 inhibitors. By correlating molecular fingerprints with bioactivity data retrieved from the ChEMBL database \cite{ref6}, we seek to identify structural features governing potency and provide a predictive framework for virtual screening.

\section{Materials and Methods}

\subsection{Data Acquisition and Curation}

Experimental bioactivity data was sourced from the ChEMBL database (version 34), a manually curated repository of bioactive molecules \cite{ref6}. We queried the database for the target beta-amyloid A4 protein (ChEMBL ID: CHEMBL2487).

The raw dataset containing 7,918 entries was subjected to the following filtration criteria:
\begin{enumerate}
    \item Bioactivity type: only inhibition assays reporting IC50 (half maximal inhibitory concentration) were retained.
    \item Units: data were standardized to nanomolar units.
    \item Data integrity: entries with missing values for standard value or canonical SMILES were removed.
    \item Duplicates: duplicate SMILES strings were handled to ensure a unique set of compounds.
\end{enumerate}

The final curated dataset consisted of 1,319 unique compounds.

\subsection{Data Preprocessing and Labeling}

To facilitate both classification and regression analysis, compounds were labeled based on their IC50 values:
\begin{itemize}
    \item Active: IC50 $\leq$ 1,000 nM (potent inhibitors)
    \item Intermediate: 1,000 nM $<$ IC50 $<$ 10,000 nM
    \item Inactive: IC50 $\geq$ 10,000 nM (weak or non-binders)
\end{itemize}

For regression modeling, the skewed IC50 values were converted to the negative logarithmic scale (pIC50) using the formula:
\begin{equation}
\mathrm{pIC50} = -\log_{10}(\mathrm{IC50} \times 10^{-9})
\end{equation}

This transformation normalizes the data distribution, making it suitable for machine learning algorithms.

\subsection{Feature Engineering}

Two types of molecular descriptors were calculated:

\subsubsection{Lipinski Descriptors}
To assess drug-likeness, we computed molecular weight, LogP (octanol-water partition coefficient), number of hydrogen bond donors, and number of hydrogen bond acceptors using the RDKit library \cite{ref7}. These descriptors allow evaluation against Lipinski's rule of five \cite{ref8}.

\subsubsection{Molecular Fingerprints}
The two-dimensional chemical structures (SMILES) were converted into numerical vectors using the PaDEL-Descriptor software \cite{ref9}. We utilized PubChem fingerprints, which consist of 881 binary bits representing the presence or absence of specific substructures.

To reduce high-dimensionality and noise, a variance threshold filter was applied. Features with low variance (threshold = 0.16) were removed, reducing the feature space from 881 to 178 informative features.

\subsection{Machine Learning Pipeline}

The dataset was split into training (80\%, $n = 1{,}055$) and testing (20\%, $n = 264$) sets. We employed the LazyPredict library \cite{ref10} to screen over 35 regression algorithms simultaneously. Based on the initial screening, the top-performing models were further optimized:
\begin{itemize}
    \item Random forest regressor: an ensemble method using bagging \cite{ref11}.
    \item Histogram gradient boosting regressor: a histogram-based gradient boosting algorithm optimized for large datasets.
    \item Support vector regression: evaluated for its robustness against overfitting.
\end{itemize}

Hyperparameter tuning was conducted using grid search cross-validation with five folds to maximize the coefficient of determination ($R^2$) \cite{ref12}.

\section{Results}

\subsection{Exploratory Data Analysis}

Chemical space analysis revealed distinct distributions for active and inactive compounds. A Mann-Whitney U test confirmed statistically significant differences ($p < 0.05$) across all Lipinski descriptors:
\begin{itemize}
    \item pIC50: active compounds showed significantly higher values ($p < 0.001$).
    \item Molecular weight: active compounds tended to have higher molecular weights ($p < 0.001$).
    \item LogP: active compounds exhibited distinct solubility profiles ($p < 0.05$).
    \item Hydrogen bond donors: significant difference ($p < 0.001$).
    \item Hydrogen bond acceptors: significant difference ($p < 0.001$).
\end{itemize}

\subsection{Model Performance}

The benchmarking results indicated that tree-based ensemble methods outperformed linear and distance-based algorithms. The performance metrics for the top models on the independent test set are summarized in Table~\ref{tab:results}.

\begin{table}[htbp]
\caption{Performance Comparison of Top Regression Models}
\label{tab:results}
\centering
\begin{tabular}{@{}lccc@{}}
\toprule
Model & Train $R^2$ & Test $R^2$ & RMSE \\
\midrule
Random forest & 0.95 & 0.78 & 0.58 \\
Histogram gradient boosting & 0.94 & 0.78 & 0.59 \\
XGBoost regressor & 0.97 & 0.77 & 0.61 \\
Support vector regression & 0.86 & 0.72 & 0.67 \\
\bottomrule
\end{tabular}
\end{table}

\subsection{Overfitting Analysis}

While random forest and histogram gradient boosting achieved the highest predictive accuracy ($R^2 = 0.78$), a gap between training ($R^2 \approx 0.95$) and testing scores suggested moderate overfitting. In contrast, support vector regression showed lower overfitting (training $R^2 = 0.86$, test $R^2 = 0.72$) but failed to capture the complexity of the data as effectively as the ensemble tree methods.

\section{Discussion}

The study successfully established a computational pipeline for identifying beta-amyloid A4 inhibitors. The high $R^2$ value (0.78) suggests that the selected 178 PubChem fingerprint features capture sufficient structural information to predict biological activity.

The significance of Lipinski descriptors confirms that bioactivity is not solely dependent on specific binding motifs but also on physicochemical properties like lipophilicity and size, which influence a molecule's ability to cross the blood-brain barrier \cite{ref13}.

Limitations of this study include the moderate overfitting observed in tree-based models, likely due to the high dimensionality of the fingerprint data relative to the sample size ($n = 1{,}319$). Future work could employ more aggressive feature selection techniques (e.g., recursive feature elimination) or deep learning architectures (e.g., graph neural networks) to improve generalization \cite{ref14}.

\section{Conclusion}

We developed robust QSAR models capable of predicting the potency of Alzheimer's drug candidates with high accuracy. The random forest and histogram gradient boosting models emerged as the most effective tools for this task. This computational framework allows for the rapid virtual screening of large chemical libraries, significantly reducing the time and cost associated with early-stage drug discovery for Alzheimer's disease.

\section*{Data Availability}

All code, data, and trained models are available in the accompanying Jupyter notebooks at the project repository.

\section*{Conflicts of Interest}

The authors declare no conflicts of interest.

\begin{thebibliography}{14}

\bibitem{ref1}
Alzheimer's Disease International, ``World Alzheimer report 2024: Global changes in attitudes to dementia,'' 2024. [Online]. Available: https://www.alzint.org/

\bibitem{ref2}
Alzheimer's Association, ``2025 Alzheimer's disease facts and figures,'' \emph{Alzheimer's \& Dementia}, vol. 21, no. 4, pp. 1598--1695, 2025.

\bibitem{ref3}
D. J. Selkoe and J. Hardy, ``The amyloid hypothesis of Alzheimer's disease at 25 years,'' \emph{EMBO Mol. Med.}, vol. 8, no. 6, pp. 595--608, 2016.

\bibitem{ref4}
Deloitte, ``Measuring the return from pharmaceutical innovation 2024,'' Deloitte Centre for Health Solutions, 2024.

\bibitem{ref5}
A. Cherkasov \emph{et al.}, ``QSAR modeling: Where have you been? Where are you going to?'' \emph{J. Med. Chem.}, vol. 57, no. 12, pp. 4977--5010, 2014.

\bibitem{ref6}
B. Zdrazil \emph{et al.}, ``The ChEMBL database in 2023: A drug discovery platform spanning multiple bioactivity data types and time periods,'' \emph{Nucleic Acids Res.}, vol. 52, no. D1, pp. D1180--D1192, 2024.

\bibitem{ref7}
G. Landrum, ``RDKit: Open-source cheminformatics software,'' 2024. [Online]. Available: https://www.rdkit.org/

\bibitem{ref8}
C. A. Lipinski, F. Lombardo, B. W. Dominy, P. J. Feeney, and A. D. Adler, ``Experimental and computational approaches to estimate solubility and permeability in drug discovery and development settings,'' \emph{Adv. Drug Deliv. Rev.}, vol. 23, no. 1--3, pp. 3--25, 1997.

\bibitem{ref9}
C. W. Yap, ``PaDEL-descriptor: An open source software to calculate molecular descriptors and fingerprints,'' \emph{J. Comput. Chem.}, vol. 32, no. 7, pp. 1466--1474, 2011.

\bibitem{ref10}
S. R. Pandala, ``LazyPredict,'' PyPI, 2022. [Online]. Available: https://pypi.org/project/lazypredict/

\bibitem{ref11}
L. Breiman, ``Random forests,'' \emph{Mach. Learn.}, vol. 45, no. 1, pp. 5--32, 2001.

\bibitem{ref12}
F. Pedregosa \emph{et al.}, ``Scikit-learn: Machine learning in Python,'' \emph{J. Mach. Learn. Res.}, vol. 12, pp. 2825--2830, 2011.

\bibitem{ref13}
W. M. Pardridge, ``Drug transport across the blood--brain barrier,'' \emph{J. Cereb. Blood Flow Metab.}, vol. 32, no. 11, pp. 1959--1972, 2012.

\bibitem{ref14}
Y. Cha, R. I. Erickson, D. L. Bhatt, and A. P. Bhatt, ``Navigating the frontiers of machine learning in neurodegenerative disease therapeutics,'' \emph{Pharmaceuticals}, vol. 17, no. 2, p. 158, 2024.

\end{thebibliography}

\end{document}
